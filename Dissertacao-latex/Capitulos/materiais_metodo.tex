% !TeX encoding = UTF-8

\chapter{MATERIAIS E MÉTODOS}\label{ch:materiais-metodos}
Conforme mencionado anteriormente, o objetivo principal dessa dissertação é especificar e aplicar um modelo de RNA para análise e predição de valores acionários. Neste capítulo será realizada a pergunta de pesquisa e dos objetivos do trabalho na escolha da metodologia. Em seguida, esta metodologia será descrita e classificada quanto ao seu conteúdo e quanto aos métodos empregados na coleta e análise de dados.

\section{IDENTIFICAÇÃO DA PESQUISA}

Para se escolher a metodologia de pesquisa, alguns aspectos importantes precisaram ser analisados. Uma pesquisa pode ser realizada com dados criados ou com dados existentes. No primeiro caso, os dados são coletados após uma intervenção destinada a provocar uma mudança. No segundo, os dados estão presentes na situação em estudo e o pesquisador, por meio das técnicas de pesquisa, faz tais dados aparecerem, sem a intenção de modificá-los, através de uma intervenção. Na pesquisa experimental, delimita-se o fenômeno, formulam-se hipóteses, determinam-se os métodos e submete-se o fenômeno à experimentação em condições de controle \cite{laville}.

A escolha de uma estratégia de pesquisa tem que ser feita considerando-se, entre outros elementos, a natureza da questão da pesquisa, o contexto no qual a pesquisa se realizará, a formação e a experiência do pesquisador. De maneira geral, pode-se dividir as pesquisas em quantitativas ou experimentais e qualitativas \cite{laville}.

Portanto, de acordo com as afirmativas acima, a presente pesquisa é caracterizada por ser experimental e qualitativa. Segundo \citeonline{chizzotti}, na pesquisa experimental, o pesquisador parte de um estado delimitado a priori, sobre qual cria possíveis hipóteses que podem ser analisadas, determinando os métodos de verificação a serem utilizados, através dos quais procurará controlar as condições do experimento.

\section{SELEÇÃO DAS AÇÕES}
Para iniciar o processo de desenvolvimento deste trabalho, é necessário definir quais ações serão escolhidas para compor a implementação. Os critérios para definir quais serão selecionadas, são caracterizados pelo alto potencial de movimentação e lucratividade na NASDAQ, ou seja, são as empresas que mais influênciam na ecônomia, que, por coincidência, são do ramo da tecnologia \cite{christie}.

Tendo em vista os critérios especificados acima, as empresas selecionadas e o código de suas ações para a realização da coleta dos dados são:
\begin{itemize}
	\item Apple (AAPL);
	\item Amazon (AMZN);
	\item Cisco Systems (CSCO);
	\item Intel (INTC);
	\item Microsoft (MSFT);
\end{itemize}

\section{TÉCNICA PARA COLETA DE DADOS}
Com a definição da bolsa de valores em questão e das empresas que serão utilizadas para realizar a implementação deste trabalho, a próxima etapa se dá pela necessidade em coletar seus dados histórios.

Como os dados históricos das ações caracterizam-se por serem séries temporais, o tipo de técnica aplicada para conseguir indicadores de relevância para uma boa análise dos objetivos propostos, refere-se à análise técnica. Portanto, esta fase de coleta de dados é muito importante para o bom funcionamento da RNA que será criada, tornando-se essencial para o êxito final do trabalho.

\subsection{Definição das Séries}
Selecionar os dados corretos para trabalhar com RNAS é uma etapa de grande relevância, pois são esses dados que serão treinados pela mesma e que irão garantir a sua estabilidade e capacidade de generalização.

Sendo assim, a partir do conhecimento concebido pelo Capítulo 3, pode-se destacar dois aspectos importantes para a análise de quais dados coletar dentro de um problema em geral, sendo eles:

\begin{enumerate}
\item Dados que influeciam diretamente no resultado final e na coêrencia da análise proposta;
\item Uso de múltiplas séries de um mesmo dado para que a rede não se sujeite à \textit{overfitting} ou \textit{underfitting}.
\end{enumerate}

Segundo \citeonline{tororadar}, as variáveis que influenciam em um sistema de análise técnica das ações são:

\begin{enumerate}
\item Abertura: Valor da ação no ínicio do dia;
\item Máximo: Valor máximo negociado no dia;
\item Mínimo: Valor mínimo negociado no dia;
\item Volume: Número de negociações do papel no dia (valor bruto das negociações);
\item Valor de fechamento no dia;
\item Médias móveis simples: As médias móveis suavizam os dados de preços para formar um indicador de tendência sequencial. Elas não prevêem a direção dos preços, mas, antes, definem a sua direção atual com um atraso. Como o próprio nome indica, uma média móvel é uma média que se move. Os dados antigos são retirados, a medida que dados mais recentes se tornam disponíveis, isto faz com que a média se mova ao longo do tempo. A maioria das médias móveis são baseadas em preços de fechamento.
\item \textit{Moving Average Convergence Divergence} (MACD): Indicador que controla a divergência e convergência das médias móveis. O MACD é formado pela diferença entre duas médias móveis, sendo uma de longo prazo e outra de curto prazo. Também, segundo \citeonline{tororadar}, as médias móveis utilizadas para o cálculo do MACD não possuem restrições, desde que, os valores de curto e longo prazo sejam respeitados. Portando, para o presente trabalho, será calculado o MACD referente às médias de 10 e 26 dias, de longo e curto prazo, respectivamente.
\end{enumerate}

\section{FERRAMENTAS PARA COLETA DE DADOS}
Com a especificação das séries que serão coletadas, é necessário preparar um ambiente que dispõe o acesso à esses dados de forma simplificada. 

Pensando nesse contexto, é essencial utilizar uma linguagem de programação que tenha suporte à esse modelo de busca por dados. Tendo em vista esta premissa, a disponibilidade de uma API ou biblioteca é de extrema importância para a automatização deste processo.

\subsection{Linguagem de Programação Python}
Segundo \citeonline{python-science}, apesar das diversas linguagens de programação disponíveis hoje para o desenvolvimento de aplicações e \textit{scripts}, a linguagem de programação Python vem se destacando e crescendo constantemente na comunidade científica, seja por sua variedade de bibliotecas que auxiliam os desenvolvedores e pesquisadores à resolverem seus problemas de forma mais simplificada, até seu poder computacional e suas características de multiplataforma e multiparadigma.

Um fator que determina a utilização da linguagem de programação Python, como dito no paragrafo anterior, se dá pela sua variabilidade de bibliotecas voltadas, principalmente, para a análise de dados. Colocando isso no contexto deste trabalho, é essencial o uso de uma ferramenta que proporcione um alto nível de funcionalidades disponíveis para serem exploradas.

\subsection{Biblioteca Pandas}\label{pandas-lib}
A biblioteca pandas é um pacote disponível na linguagem de programação Python que fornece fácil acesso à dados disponíveis na internet. O objetivo principal do pandas é, justamente, fornecer buscas flexíveis e intuitivas para a análise de dados estruturados e não estruturados. Além disto, busca como objetivo específico, se tornar a ferramenta de análise, busca e manipulação de dados, aberta, mais poderosa e flexível existente \cite{pandas}.

Segundo \citeonline{pandas}, o seu uso é adequado para diferentes tipos de dados, sendo eles:
\begin{enumerate}
\item Dados tabulares com colunas, como em uma tabela \textit{Structured Query Language} (SQL) de um banco de dados relacional ou uma planilha do Excel;
\item Dados de séries temporais não ordenados;
\item Qualquer outra forma de conjuntos de dados observacionais e estatísticos.
\end{enumerate}

Portanto, a principal justificativa para a utilização da biblioteca se dá pelo seu desenvolvimento, de código aberto, que impulsiona a comunidade a realizar diferentes aplicações, além da sua excelente documentação e suporte de acesso aos dados históricos da bolsa de valores NASDAQ, que são requisitos para a realização deste trabalho.

Para exemplificar como a biblioteca facilita o processo de coleta dos dados, o Código 2 ilustra uma pesquisa das ações da Microsoft, no período de 05/03/2001 a 11/04/2001.
\codigoPython\
\lstinputlisting[language=Python, label=script-coleta-dados, caption=\textit{Script} para coleta de dados]{src/microsoft.py}

As séries históricas são disponibilizadas em arquivos com valores separados por vírgulas (*.csv), contendo a data de negociação, valor de abertura, valores máximo e mínimo atingidos, valor de fechamento e o volume de negociações. Este formato de de arquivo, segundo \citeonline{pandas}, é chamado de \textit{DataFrame}.

Na Figura 10 é possível observar o arquivo criado a partir do Código 2, demonstrando o formato que foi especificado no código e o que a API retorna para a manipulação.

\begin{figure}[h]
	\centering
	\fbox{\includegraphics[width=0.5\textwidth, height=0.33\textheight]{dados-pandas}}
	\caption{Representação dos dados coletados pela biblioteca pandas}
	\fonte{Elaborado pelo autor}
	\label{exec-coleta-dados}
\end{figure}

\subsection{Google \textit{Finance} API}
É importante citar qual a fonte onde o pandas busca as séries históricas, por questões como confiabilidade dos dados. O acesso é realizado através da integração com o Google \textit{Finance} API, um banco de dados quase em tempo real mantido pelo Google, que trabalha com as principais bolsas de valores do mundo \cite{pandas}.

A API trabalha com um limite inicial para a coleta das ações, a partir de 03/05/2001 e vai se atualizando diariamente. Tendo em vista que para trabalhar com RNAs é importante buscar o máximo de informações possíveis e, também, considerando o limite disponível pela API, o período das ações das empresas selecionadas serão coletadas iniciando no ano de 2001 até o ano de 2017. 

Sendo assim, a ideia é preparar um ambiente que proporcione o uso dessa ferramenta, junto à biblioteca citada na \autoref{pandas-lib} para realizar as buscas pelos dados necessários.

\subsection{PyCharm IDE}
É necessário utilizar uma \textit{Integrated Development Environment} (IDE) para auxiliar no \textit{script} de coleta de dados. PyCharm é uma IDE proprietária do grupo JetBrains que conta com uma versão livre para estudantes, que será realizada neste trabalho.
 
PyCharm suporta o desenvolvimento com a linguagem de programação Python desde a sua versão 2.4 até a versão 3.6. Vale citar que outras IDEs para o desenvolvimento poderiam ser utilizadas, porém a praticidade e a boa documentação das ferramentas JetBrains são fatores que justificam a sua utilização \cite{pycharm}.

\section{ESPECIFICAÇÃO DO MODELO DE REDES NEURAIS ARTIFICIAIS}\label{especifica-modelo}
Tendo como base o conhecimento adquirido no Capítulo 3, a definição de um modelo de RNA varia muito, sendo específico para cada problema do mundo real. Por isso, é importante frisar que não existe um modelo pré-definido e eficiente em RNAs, ficando a cargo da análise dos resultados, e, muitas vezes, por métodos de tentativa e erro para se chegar à um resultado eficiente \cite{haykin2000}.
 
Porém, com o estudo realizado no Capítulo 3, esse processo pode ser menos custoso do que parece. Avaliar o problema em que se está trabalhando, junto com o conhecimento teórico, é de grande auxílio para a especificação e construção de um modelo inicial que se aproxime do desejado.

Para especificar quais modelos não devem ser aplicados, ou seja, modelos que não tenham características referente a resolução do problema em questão, algumas perguntas-chave serão elaboradas, sendo elas:

\begin{enumerate}\label{enumerate-questions}
\item "Qual o modelo de dados se está trabalhando?";
\item "Estes dados serão treinados a partir de resultados esperados?";
\item "A RNA será exposta à uma quantidade significativa de dados desconhecidos aos treinados?".
\end{enumerate}

A primeira pergunta refere-se ao modelo de transmissão dos dados que a RNA deve operar, ou seja, se a mesma será formada por camadas adiante ou recorrente. Como no presente trabalho serão analisados dados através de séries temporais, segundo \citeonline{haykin2000}, a RNA deverá trabalhar com um modelo de camadas alimentada adiante (\textit{Feedforward}).

A segunda pergunta refere-se ao processo de aprendizagem que deve ser utilizado. Se os dados são treinados a partir de resultados esperados, a RNA deverá possuir um método de aprendizado supervisionado, caso contrário, um modelo de RNA com treinamento não supervisionado deverá ser utilizado. Para este trabalho, a RNA será treinada a partir dos resultados esperados através das séries temporais, ou seja, o modelo trabalhará com aprendizagem supervisionada.

A terceira pergunta refere-se à capacidade de generalização. Sendo assim, um modelo de RNA que irá trabalhar com dados desconhecidos deve ter caracteristícas de boa generalização, caso contrário, um modelo simples já é o suficiente.

\subsection{Definição da Arquitetura}

Tendo em vista os critérios estabelecidos na seção anterior, chega-se a definição de que a RNA deverá possuir uma arquitetura de camadas alimentada adiante, um método de aprendizado supervisionado e uma função de custo, para controlar o processo de generalização da mesma.

\citeonline{haykin2000} direciona este modelo de rede ao \textit{Perceptron} de Múltiplas Camadas (MLP, \textit{Multilayer Perceptron}), são redes neurais compostas por um conjunto de unidades sensoriais que constituem a camada de entrada, uma ou mais camadas ocultas e uma camada de saída, onde o sinal de entrada se propaga para frente através da rede, camada por camada. Redes MLP trabalham normalmente junto ao algoritmo \textit{backpropagation} para realizar seu treinamento e aprendizado.

Tendo em vista a especificação do modelo da rede e dos dados que serão utilizados para realizar o treinamento e os testes, a próxima etapa será definir quais serão as entradas utilizadas. Como o objetivo do trabalho é realizar a predição do valor de abertura das ações em dias posteriores aos treinados, todas as váriaveis coletadas serão utilizadas como entrada da RNA. Portanto, os parâmetros que irão compor a camada de entrada da RNA são:

\begin{enumerate}
\item Valor de abertura;
\item Valor máximo negociado no dia;
\item Valor mínimo negociado no dia;
\item Valor de fechamento;
\item Número de negociações do papel;
\item Média móvel simples de 10 dias;
\item Média móvel simples de 26 dias;
\item \textit{Moving Average Convergence Divergence} (MACD).
\end{enumerate}

Segundo \citeonline{haykin2000} e \citeonline{marangoni}, na camada intermediária da RNA existe um cuidado especial para especificar sua composição, pois o número de neurônios depende de várias condições, tais como:

\begin{itemize}
\item Quantidade de dados para treinamento em relação a quantidade de neurônios na camada oculta: Deve-se ter cuidado para não utilizar neurônios e camadas demais, o que pode levar a rede a memorizar os padrões de treinamento, quebrando seu processo de aprendizagem (\textit{overfitting}). O inverso também é válido, poucos neurônios para uma grande quantidade de dados, pode levar a rede a não encontrar uma solução desejável, pelo pouco poder de processamento (\textit{underfitting});
\item Definir o número de neurônios em função da dimensão das camadas de entrada e saída da rede: Pode-se definir o número de neurônios na camada escondida como sendo a média aritmética entre o tamanho da entrada e da saída da rede;
\item Utilizar um número de sinapses dez vezes menor que o número de exemplos disponíveis para treinamento.
\end{itemize}

Por isso, para a grande maioria dos problemas utiliza-se apenas uma camada escondida, onde sua quantidade de neurônios é definida buscando um meio termo entre a quantidade de entradas e a sua saída, tentando evitar as complicações evidenciadas nos tópicos anteriores.

Mediante estas informações, o modelo será desenvolvido com apenas uma camada intermediária composta por 13 neurônios, buscando otimizar a sua estrutura e obtendo um resultado eficiente para o problema.

Por fim, a saída da rede será composta por uma única camada que irá conter o valor de abertura do dia seguinte. É importante deixar claro que este valor, durante o processo de treinamento, será preenchido com os valores reais das ações em relação ao período que será escolhido para o processo de treinamento.

Na Figura 11 é possível visualizar a composição gráfica da arquitetura proposta.

\begin{figure}[h]
	\centering
	\fbox{\includegraphics[width=0.8\textwidth, scale=0.5]{arquitetura-rna}}
	\caption{Modelo de arquitetura proposta utilizando o algoritmo \textit{backpropagation}}
	\fonte{Elaborado pelo autor}
	\label{exec-rna}
\end{figure}

\section{DEFINIÇÃO DA BASE DE TREINAMENTO}
A partir do modelo de rede definido, é necessário classificar os dados que serão utilizados na fase de treinamento e os que serão utilizados para a realização dos testes. Como no trabalho serão coletadas as ações de 5 empresas distintas, a ideia é que para cada empresa, individualmente, sejam realizados os processos de treinamento e testes, com objetivo de observar o comportamento da rede nas mais diversas situações e proporcionar uma análise mais detalhada do modelo.

Os dados utilizados para o treinamento da rede devem ser significativos e cobrir amplamente o domínio do problema, e não apenas envolver os casos de operações normais ou rotineiras, mas também incorporar casos de exceção e de situações limites. Seguindo o funcionamento do algoritmo \textit{backpropagation}, os dados treinados devem seguir os seguintes passos:

\begin{enumerate}
\item Aplicar o vetor de entrada da rede;
\item Calcular a saída da rede;
\item Ajustar os pesos da rede de maneira a minimizar o erro;
\item Repetir o passo (1) até o passo (3) para cada vetor do conjunto de treinamento, até que o erro (função de custo) seja satisfeito. 
\end{enumerate}

Segundo \citeonline{haykin2000}, ao determinar a classificação de um conjunto para treinamento, devem ser levados em consideração os seguintes fatores:

\begin{itemize}
\item Quando há uma grande quantidade de amostras disponíveis para o treinamento, a divisão deve se aproximar de 33\% para treino e 67\% para testes;
\item Quando o número de amostras é reduzido deve-se diminuir a proporção do conjunto de testes, com um valor próximo de 60\% para treinamento e 40\% para testes;
\item Em casos com um número muito pequeno de amostras, é recomandável utilizar 50\% para ambos.
\end{itemize}

\section{BIBLIOTECA PYBRAIN}
\textit{Python-Based Reinforcement Learning, Artificial Intelligence and Neural Network Library} (PyBrain) é uma biblioteca de aprendizagem de máquinas para Python. Seu objetivo é oferecer algoritmos flexíveis e fáceis de usar que proporcione uma forma mais intuitiva de desenvolver algoritmos complexos relacionados à Inteligência Artificial. Embora existam poucas bibliotecas de aprendizado de máquinas, PyBrain pretende ser uma biblioteca modular fácil de aprender, podendo ser utilizada por estudantes de nível básico até um nível mais avançado, oferecendo flexibilidade e algoritmos para pesquisa de ponta \cite{pybrain2010}.

Segundo \citeonline{pybrain2010}, como o próprio nome sugere, PyBrain contém algoritmos para redes neurais, aprendizagem supervisionada, aprendizagem não supervisionada e algoritmos evolutivos. A biblioteca é construída com seu núcleo baseado em redes neurais, onde todos os métodos de treinamento aceitam uma rede como instância pré-treinada. Isso faz de PyBrain uma ferramenta poderosa para a execução de tarefas complexas de serem resolvidas. 

A Figura 12 demonstra, de maneira mais clara, como a biblioteca PyBrain irá auxiliar no projeto.

\begin{figure}[h]
	\centering
	\fbox{\includegraphics[width=0.8\textwidth, scale=0.5]{pybrain8}}
	\caption{Arquitetura da biblioteca PyBrain aplicada na estrutura do trabalho}
	\fonte{Adaptado de \citeonline{pybrain2010}}
	\label{exec-rna-pybrain}
\end{figure}

PyBrain tem ênfase nas arquiteturas de rede, que podem ser treinadas e manipuladas com quase todos os algoritmos oferecidos de aprendizado de máquina. As redes são compostas por módulos que estão ligados através de conexões, onde os nós são os módulos e as bordas são as conexões. Isso torna o PyBrain muito flexível, mas também necessário em todos os casos, seguir este padrão.

Alguns exemplos de arquiteturas disponibilizada pela biblioteca são:

\begin{itemize}
	\item Redes Neurais Alimentadas Adiante (\textit{Feedforward});
	\item Redes Neurais Recorrentes;
	\item Redes Neurais Recorrentes Multidimensionais;
	\item Redes de Kahonen (Mapas Auto-Organizáveis);
	\item Redes Bidirecionais;
	\item Topologias Personalizadas.
\end{itemize}

Dentre os algoritmos de aprendizagem supervisionada utilizado pelo PyBrain, destacam-se:

\begin{itemize}
	\item Algoritmo de retropropagação (\textit{backpropagation});
	\item Algoritmo de R-prop;
	\item Algoritmos de Máquinas de Vetores de Suporte (SVM).
\end{itemize}	

Todos os algoritmos e implementações matemáticas da PyBrain tem como base o SciPy. SciPy é um ecossistema baseado em Python, de código aberto, para matemática, ciência e engenharia. Sua biblioteca central é o NumPy, que fornece uma manipulação conveniente e rápida de estruturas de dados que armazenam os valores em mais de uma dimensão. Portanto, a biblioteca SciPy foi desenvolvida para trabalhar com \textit{arrays} NumPy, fornecendo rotinas amigáveis e eficientes para realizar cálculos complexos \cite{scipy}.

Também é importante evidenciar que a biblioteca PyBrain é de código aberto e gratuita para todos, licenciada sob a \textit{Berkeley Software Distribution} (BSD) Software.

\section{BIBLIOTECA MATPLOTLIB}
O matplotlib é uma biblioteca disponível para a linguagem de programação Python que contém recursos para a geração de gráficos 2D, a partir de um determinado conjunto de dados. Seus gráficos são criados com alta qualidade utilizando comandos simples e intuitivos \cite{matplotlib}.

Segundo \citeonline{matplotlib}, o código Matplotlib é dividido conceitualmente em três partes:
\begin{itemize}
\item pylab: conjunto de funções disponíveis em matplotlib.pylab que permite a escrita de códigos simples de serem criados;
\item API: camada responsável por realizar toda a lógica da biblioteca, criando por exemplo figuras, textos, gráficos e linhas;
\item \textit{backend}: conjunto de funções que dependem do dispositivo de saída, como por exemplo imagens e arquivos no formato \textit{Portable Document Format} (PDF).
\end{itemize}

Desta forma, todos os métodos necessários para realizar a implementação deste trabalho foram levantados, o próximo capítulo irá detalhar o processo de desenvolvimento das técnicas, e, consequentemente, da obtenção dos objetivos desejados, levando todos os aspectos estudados e definidos até o momento considerado.