% !TeX encoding = UTF-8

\chapter{ANÁLISE DOS RESULTADOS}\label{ch:resultados}
Este capítulo tem como finalidade apresentar os resultados obtidos através das implementações demonstradas no Capítulo 5.

\section{ANÁLISE DAS SÉRIES UTILIZADAS}
Após a implementação do modelo da RNA, através dos conjuntos de dados coletados, faz-se necessário avaliar como a respectiva técnica se comportou. Tendo isso em vista, a mesma foi aplicada em função do objetivo principal do trabalho, que é medir a capacidade de precisão de acerto no valor de abertura das ações. Portanto nas seções posteriores foi elaborada, de forma independente, uma análise de eficácia do modelo para o cenário de cada empresa utilizada no presente trabalho. É importante ressaltar que os modelos foram implementados levando em consideração todos os \textit{scripts}, métodos e ferramentas utilizadas no capítulo anterior.

\subsection{Análise do modelo da RNA: Intel Corporation}
A rede da Intel Corporation foi treinada com o objetivo de capturar o maior nível de variação possível dos dados, visando mapear um maior conjunto de padrões e, assim, responder de forma eficiente à dados dispersos através de uma boa capacidade de generalização. Tendo isso em vista, o período de treinamento preparado foi de 09/04/2001 até 21/08/2017.

Para o presente caso foram construídos dois cenários de simulação, buscando capacitar a rede à utilizar um parâmetro de ciclos de treinamento adequado ao modelo de dados utilizado. As métricas definidas para análise, a partir destes ciclos de treinamento, são: o comportamento da função de custo que compõem o modelo (erro quadrático médio, EQM) e a margem de erro dos valores resultantes da rede no período de 23/08/2017 a 31/08/2017 em relação aos valores reais. Os parâmetros utilizados foram: 200 e 1000 iterações.

\subsubsection{Corporation: Treinamento com 200 Iterações}	
A ideia de executar o treinamento com uma quantidade baixa de iterações, é feita com o intuito de proporcionar um treinamento mais rápido com resultados significativos, levando em consideração as características especificas do modelo de dados. Assim, a configuração de treinamento da rede foi de 200 iterações x 4117 linhas, onde as linhas são a quantidade de registros do período de testes. Portanto, o modelo foi exposto a calculado por um total de 823.400 exemplos. O Gráfico 8 demonstra, de acordo com a quantidade de ciclos, a variação do EQM.
\begin{grafico}[h]
	\centering
	\fbox{\includegraphics[width=1\textwidth]{erro_intel_comgrid}}
	\caption{Decaimento do EQM no treinamento da rede}
	\fonte{Elaborado pelo autor}
	\label{lingua}
\end{grafico}

O Gráfico 8 ilustra a variação dos resultados obtidos. O EQM iniciou-se, na primeira iteração, com um erro percentual de 0.00248996652176 e, após o término das 200 épocas de treinamento, concluí-se com um erro percentual de 9.49359025338e-05.

Analisando o gráfico, pode-se observar que o erro obteve uma queda brusca nas primeiras 75 iterações, enquanto que, nas 125 iterações posteriores, manteve o padrão com valores aproximados. A queda brusca e rápida do valor está diretamente ligada ao modelo de dados utilizado. Como detalhado na Seção 5.1, as ações da Intel não sofrem grande variabilidade no período coletado, tornando-se assim mais rápida de ser abstraída pela RNA.

Já o período destinado à testes, especificado na Seção 6.1.1, é exposto de na Tabela 1, detalhando quais foram os valores aplicados.
\begin{table}[h]
\centering
\caption{Um nome qualquer}
\vspace{0.5cm}
\begin{tabular}{llllll}  
\toprule
Data    & Abertura   & Alta   & Baixa   & Fechamento   & Volume\\
\midrule
23/08/2017 & 34.54 & 34.81 & 34.38 & 34.66 & 196.481,34\\
24/08/2017 & 34.70 & 34.89 & 34.55 & 34.71 & 143.018,92\\
25/08/2017 & 34.82 & 34.93 & 34.58 & 34.67 & 147.268,29\\
28/08/2017 & 34.78 & 34.80 & 34.59 & 34.65 & 207.128,76\\
29/08/2017 & 34.51 & 34.75 & 34.46 & 34.73 & 158.436,68\\
30/08/2017 & 34.75 & 34.96 & 34.63 & 34.89 & 185.650,07\\
31/08/2017 & 34.94 & 35.18 & 34.87 & 35.07 & 163.667,72\\
\bottomrule
\end{tabular}
\end{table}

Os dados demonstrados na Tabela 1 foram refinados e normalizados, de acordo com as especificações e cálculos realizados no Capítulo 5. Após realizado este processo, os mesmos foram inseridos na RNA para realizar a predição. Posteriormente, foi realizada a desnormalização e os resultados apresentados. A Tabela 2 ilustra os resultados obtidos.




